\fancyhead[LE,RO]{Partition equilibrium of I$_2$ between two phases -- ,,JOD''}
\fancyhead[LO,RE]{\thesection}
\fancyfoot[LE,RO]{\thepage}
\fancyfoot[RE,LO]{\emph{Physical chemistry lab. practice for pharmacy students}}

\setcounter{section}{7}
\section{Partition equilibrium of I$_2$ between two phases}
\subsection{Introduction}
If a two

Ha két, egymással nem elegyedő oldószerben egy anyag egyidejűleg oldódik, és a két oldószer érintkezésbe kerül egymással, az egyensúly beállása után az oldott anyagnak a két oldószerben érvényes aktivitásának (koncentrációjának) aránya állandó.
Egy ilyen, két fázisból (A és B) és három komponensből álló rendszernek a Gibbs-féle fázisszabály szerint három szabadsági foka van: a nyomás és hőmérséklet mellett a megoszló anyag koncentrációja az egyik fázisban.
A termodinamikai egyensúly adott hőmérsékleten és nyomáson - a megoszlást illetően - akkor áll be, mikor a megoszló anyag (x) kémiai potenciálja mindkét fázisban azonos lesz:
ux,A = ux,B
u*x,A + RT ln ax,A = u*x,B + RT ln ax,B
ahol u*x,i - a megoszló anyag standard kémiai potenciálja az i-edik fázisban
ax,i - az x-anyag aktivitása az i-edik fázisban
Az egyenletet átrendezve
(u*x,A - u*x,B) / R x T = ln ax,B / ln ax,A                (8.1)
látható, hogy az egyenlet bal oldala (adott hőmérsékleten és nyomáson) konstans.
Ezt megoszlási állandónak nevezzük.
(A logaritmus függvény monotonitása miatt az ax,B/ax,A hányadost is szokás használni, ezt megoszlási hányadosnak nevezzük.)
Az egyenlet értelmében a megoszlás nem függ a fázisokban mért abszolút koncentrációtól, csak ezek arányától.
Ez azonban csak egy fontos feltétel teljesülésekor igaz: nevezetesen abban az esetben, ha az oldott anyag egyik oldószerben sem szenved disszociációt, illetve nem alkot asszociátumokat sem.
Amennyiben e feltétel nem teljesül, a megoszlás igaz marad az oldott anyagra nézve, ha olyan analitikai technikával követjük koncentrációjának megváltozását, mely nem érzékeny az oldott anyagból valamely - vagy mindkét- oldószerben keletkező további - esetleg egyensúlyi- formákra.
Az elterjedt - és a gyakorlaton alkalmazott - mérési eljárások azonban olyanok, hogy csak az adott fázisban található összes oldott anyag meghatározására nyújtanak lehetőséget.
Így a fenti egyenlettől látszólagos eltérések tapasztalhatók.
Ezen eltérések azonban igen fontosak lehetnek számunkra: felvilágosítást adhatnak az oldott anyagnak az adott oldószerben kialakuló esetleges szerkezeti / disszociációs / asszociációs tulajdonságairól.
Tekintsünk egy egyszerű példát: a cA koncentrációjú A anyag a poláris fázisban legyen protonálható, így a poláris fázisbeli (v) összkoncentrációja cA,v = [A]v + [AH+]v.
Amennyiben az apoláris fázisba (o) csak az A forma extrahálható, a megoszlási állandóra nem a K = cA,o/cA,v, hanem a K = cA,0/[A]v kifejezést találjuk állandónak.
Ehhez hasonlóan a szerves fázisban is eltérő formában lehet jelen a megoszló komponens: pl. közismert, hogy organikus gyengesavak apoláris oldószerben gyakran dimereket, vagy magasabb rendű aggregátumokat képeznek.
Ilyen esetben szintén korrekciót kell alkalmaznunk, nevezetesen mindig meg kell keresnünk azt a komponenst, melyre a megoszlási egyensúly felírható, továbbá a lehetséges egyensúlyi formáinak figyelembevételével a tömeghatás törvény értelmében korrigálnunk kell az aktivitására/koncentrációjára vonatkozó összefüggéseket.

A fentiek alapján egy gyengesav vizes és szerves oldószeres fázisok között történő megoszlásakor mindkét hatással - tehát a vizes közegben bekövetkező disszociációval és a szerves fázisban történő asszociációval - is számolnunk kell.
Legyen cAH,v az AH gyengesav vizes közegben mért analitikai koncentrációja, és cAH,o a szerves fázisban mért összes koncentráció.
Minthogy vizes közegben a sav protonokra és savmaradékra disszociál, a vizes közegben lévő AH forma koncentrációja [AH] = cAH,v(1-alfa) alakban írható fel, ahol alfa a disszociáció foka.
Ez az AH forma tart egyensúlyt a szerves fázisban lévő AH molekulákkal, mivel feltételezhetjük, hogy a vízzel nem elegyedő szerves oldószer kis permittivitása miatt abban a töltött részecskék kialakulása energetikailag nem kedvezményezett folyamat.
A megoszlási hányadosra tehát:
K = cAH,v(1-alfa) / cAH,0                   (8.2)
összefüggést nyerjük.
Az (8.1) egyenlettől való eltérés abban nyilvánul meg, hogy a disszociáció foka függ a gyengesav koncentrációjától és ennek mindenkori értéke a savi disszociációs állandó (Ks) ismeretében számítható (a gyengesav igen híg vizes oldatában jó közelítéssel alfa = (Ks/cAH,v)1/2 adható meg).
A szerves fázisban ugyanakkor a gyengesav dimereket, vagy más, n részecskéből álló asszociátumokat hozhat létre (AH)n.
A tömeghatás törvénye értelmében - feltételezve hogy a szerves fázisban az [AH] kicsi – a 
Kass = [AH]n / cAH,0
egyenlet lesz érvényes.
Ekkor a megoszlási hányados -a vizes közegben történt disszociáció is figyelembe véve –
K = …….. képlet                                  (8.3)
alakban írható fel.
Ebből az összefüggésből a szerves fázisban képződő asszociátumok összetétele egyszerűen grafikusan meghatározható átrendezés és logaritmálás után:
ln(cAH,v x (1 –alfa)) = lnK + 1/n x lncAH,0                                (8.4)

Egy anyag megoszlását adott oldószer-párban tehát számos körülmény befolyásolja.
Amennyiben a megoszló anyag nem semleges molekula (pl. kvaterner ammónium-, foszfónium sók, egyéb ionos tenzidek, stb), megoszlását az ellenionok tulajdonságai is befolyásolják.
A tapasztalat szerint egy kiválasztott kvaterner ammónium vegyület esetében változtatva az anion minőségét, a szerves oldószerben történő oldhatóság ClO4 körülbelül jel SCN- > NO3 - > Br- > Cl- >> CH3COO- sorrendben csökken egyértékű ionok esetében, melyet az anionok Hoffmeister sorának nevezünk.
E jelenséget számos analitikai módszerben hasznosítják, gondoljunk csak a nitrát ion spektrofotometriás, vagy az anionos tenzidek kétfázisú titrálással történő mennyiségi meghatározására.


2. A gyakorlat leírása


a) Jód megoszlása víz - szerves oldószer rendszerben

Analitikai mérlegen mérjünk le mintegy 0.1 g jódot.
Analitikai mérlegen mérjünk le mintegy 0.1 g jódot. A gyakorlatvezető által kijelölt szerves oldószer 20 cm3-jében oldjuk fel és vigyük csiszolt dugós Erlenmeyer-lombikba az oldatot.
150 cm3 desztillált víz hozzáadása után zárjuk le az edényt és helyezzük a rázógépbe.
20 perc rázatás után töltsük a lombik tartalmát választótölcsérbe, és különítsük el a fázisokat.
Pipettázzunk egy-egy lombikba a szerves fázisból 5, a vizes fázisból 100 cm3-t a további analízis céljára.
A maradék oldatokat egyesítsük és pótoljuk ki 5 cm3 szerves oldószerrel és 100 cm3 desztillált vízzel, majd helyezzük ismét a rázógépbe 20 percre.
Ismételjük meg a kísérletet háromszor.
Az analízis céljára elkülönített oldatok jódtartalmát ismert faktorú nátrium-tioszulfáttal történő titrálással határozzuk meg.
A szerves oldószerben a végpontot a jód színének eltűnése jelzi, míg a vizes fázis titrálásakor indikátorként keményítő oldatot használunk.

A gyakorlat idejének jobb kihasználása végett az első rázatás alatt célszerű a tioszulfát mérőoldat faktorát meghatározni, majd a további rázatások alatt az analízis céljára elkülönített fázisokét.

Számítsuk ki az egyes fázisokban a jód koncentrációját mol dm-3-ben, majd ezek felhasználásával a megoszlási hányadost.
Az eredményeket az alábbi táblázat szerint adjuk meg:
tioszulfát oldat koncentrációja:
tioszulfát oldat faktora:
bemért jód mennyisége: g, mol

táblázat

Számítsuk ki a kapott megoszlási hányados értékek átlagát és azok szórását is!


Nátrium-tioszulfát mérőoldat készítése és faktorozása:

A mérőoldatot kristályos nátrium tioszulfátból (Na2S2O3 x 5 H2O, Mr = 248.2) készítjük.
Analitikai mérlegen bemérjük a 100 cm3 0.01 M oldatnak megfelelő mennyiségű sót (0.2482 g),mérőlombikba visszük, majd izobutanol 1 tf%-os vizes oldatával feltöltjük a lombikot mintegy 1 cm-re a jeltől.
A jelre állítást bidesztillált vízzel végezzük.

A faktor meghatározását KIO3 oldatra végezzük az alábbiak szerint: 100 cm3-es üvegdugós Erlenmeyer lombikban 10.00 cm3 0.0015 M KIO3 oldatot pipettázunk.
Mintegy 20 cm3 bidesztillált vizet és 1 cm3 20% sósavat adunk hozzá, majd kb. 0.3 g KI-ot oldunk benne.
Az elegyet mintegy 2 perc várakozási idő után tioszulfát mérőoldattal titráljuk.
A titrálás vége előtt szemcseppentővel néhány csepp keményítő oldatot juttatunk a titráló lombikba.
A végpontot a jódkeményítő színének eltűnése jelzi.
A reakciók egyenletei a következők:
IO3- + 5 I- + 6 H+ = 3 I2 + 3 H2O
2 S2O32- + I2 = S4O62- + 2 I
Az egyenletekből látható, hogy 1 mol jodát 6 mol tioszulfáttal ekvivalens.


b) Gyengesav megoszlása víz-szerves oldószer rendszerben

A gyakorlatvezető által kijelölt szerves oldószer 50 cm3-éhez adjunk 50 cm3 desztillált vizet, valamint a kijelölt gyengesavból 0.5 cm3-t csiszolt dugós Erlenmeyer-lombikban.
Zárjuk le az edényt és helyezzük a rázógépbe.
20 perc rázatás után töltsük a lombik tartalmát választótölcsérbe és különítsük el a fázisokat.
Pipettázzunk egy-egy lombikba a szerves fázisból 25, a vizes fázisból 10 cm3-t a további analízis céljára.
A maradék oldatokat egyesítsük, pótoljuk ki 25 cm3 szerves oldószerrel és 10 cm3 desztillált vízzel, majd helyezzük ismét a rázógépbe 20 percre.
Ismételjük meg a kísérletet háromszor.


Az analízis céljára elkülönített oldatok gyengesav-tartalmát ismert faktorú, 0.05 M nátriumhidroxid oldattal történő titrálással határozzuk meg.
A végpont jelzésére fenolftaleint használunk.

A gyakorlat idejének jobb kihasználása végett az első rázatás alatt célszerű a NaOH mérőoldat faktorát (faktorozott sósavra, metilnarancs indikátor mellett) a szokásos módon meghatározni, majd a további rázatások alatt az analízis céljára elkülönített fázisok gyengesav-tartalmát megmérni.

Számítsuk ki az egyes fázisokban a gyengesav koncentrációját mol dm-3-ben, majd ezek felhasználásával a megoszlási hányadost.
Az eredményeket az alábbi táblázat szerint adjuk meg:
gyengesav: moláris tömege: g mol-1
sűrűsége: g cm-3 Ka= mol dm-3
NaOH oldat faktora:

táblázat

Számítsuk ki a koncentrációk logaritmusait, határozzuk meg grafikusan először n értékét, majd számítsuk ki -a következő táblázatnak megfelelően- többféleképpen is a megoszlási hányados értékét.
Az eredményekhez mellékeljük az n-érték meghatározására készített ábrát is.

táblázat
