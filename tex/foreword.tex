\section{Foreword}

Today, automatization of measurements seems to reduce the role of human operators in a lot of areas of science and industry. To understand and improve these methods, however, a closer interaction with the measured phenomena is necessary. It is the authors' hope that these practices will help students to understand the fundamental aspects of measurements in physical chemistry, and the derivation of important parameters from recorded data in general.

This handout is primarily written for the physical chemistry laboratory practice of Chemistry BSc. and pharmacy students. In each practice, you will measure physical and chemical properties, and then use basic relationships in physical chemistry to calculate important parameters such as heat of dissolution, rate constant, pK, selectivity coefficient of ion-selective electrodes, and others.

During the course, the you will familiarize yourself with basic and intermediate level methods in physical chemical measurements. The authors however, assume a basic knowledge of general and analytical chemistry, regarding simple concentration calculations, titrimetric methods, basic electrochemistry, and photometric methods.

The effort on each practice should be divided into three more or less equal parts. First, it is essential for a successful practice to prepare in advance. Second, the practice itself should be carried out with great care and precision. Good laboratory practice (GPL) is advised for all students, and regardless of the topic. And third, a laboratory notebook should be prepared during, and finished after the practice to make it complete with the necessary calculations, figures, and conclusions.

The authors wish a successful course for each student undertaking these practices and hope to contribute to their laboratory skills and their understanding of physical chemistry.

\vspace{2 cm}

Pécs, 2018 December \hfill The authors
