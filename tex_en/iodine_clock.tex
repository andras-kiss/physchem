\section{Investigating the iodine clock reaction. Determination of the initial rate and reactions orders}
\subsection{Introduction}

As seen during the investigation of the first order process, the order of a reaction with respect to a selected component can be determined by the method of initial rates: the concentration of the selected component must be varied within a series of experiments while the concentrations of all others must be kept constant. Under acidic conditions, iodate and ioidide ions react in a process described by the following chemical equation:

\begin{equation}
\text{IO}_{3}^{-} + 5 \text{I}^{-} + 6 \text{H}^{+}  \longrightarrow 3 \text{I}_{2} + 3 \text{H}_{2} \text{O}
\end{equation}

This is not a simple reaction. Three different reactants are necessary, and all of them have different stoichiometric coefficients. If the reaction obeys power law kinetics, the rate law can be given in the following form:

\begin{equation}
r_0 =\frac{d[\text{IO}_{3}^{-}]}{dt} = k[\text{IO}_{3}^{-}]^{\beta \text{IO}_{3}^{-}} [\text{I}^{-}]^{\beta \text{I}^{-}} [\text{H}^{+}]^{\beta \text{H}^{+}}
\end{equation}

Brackets in this equation mean the (molar) concentration of the species enclosed.

The reaction can be monitored as follows:  the iodine produced forms a highly colored inclusion compound with starch. However, iodine is reacted with an auxiliary reactant, which is used at the same initial concentration in all experiments, but this is a lot lower than the initial  concentrations of all other recatants. This allows a low conversion for the reaction we wish to study, so the inital rate and other kinetic parameters can be determined releatively simply. As long as the auxiliary substance (arsenous acid in this particular case) is present, iodine does not accumulate but reacts further in a fast reaction. If the order of reaction with respect to iodate ion is to be determined, the initial concentration of iodate ion is varied systematically in the presence of arsenous acid. The amount of this auxiliary substance sets a constant conversion of the process at which the color of the iodine starch complex becomes visible. The color chagne is sudden and time from the mixing to the observable change can be measured easily. Iodine and arsenous acid react as follows:

\begin{equation}
\text{H}_{3}\text{AsO}_{3} + \text{I}_{2} + \text{H}_{2}\text{O}  \longrightarrow \text{HAsO}_{4}^{2-} +  2 \text{I}^{-}  + 4 \text{H}^{+}
\end{equation}

The initial concentraton of arsenous acid can also be used to control the time at which iodine appears, so this reaction is sometimes called a clock reaction.


\subsection{Practice procedures}

Prepare the following solutions:

\begin{enumerate}
\item 0.2 M KI solution (100 cm$^3$)
\item 0.1 M KIO$_3$ solution (50 cm$^3$)
\item 0.75 M NaCH$_3$COO solution (250 cm$^3$)
\item 0.2 M CH$_3$COOH solution (250 cm$^3$)
\item Buffer A: Measure 100 cm$^3$ 0.75 M NaCH$_3$COO solution and 100 cm$^3$ 0.2 M CH$_3$COOH solution into a 500 cm$^3$ volumetric flask. Fill up the flask to the mark. (This will give [H$^+$] = 1 $\times$ 10$^{-5}$ M.)
\item Buffer B: Measure 20 cm$^3$ 0.75 M NaCH$_3$COO solution and 40 cm$^3$ 0.2 M CH$_3$COOH solution into a 100 cm$^3$ volumetric flask. Fill up the flask to the mark. (This will give [H$^+$] = 2 $\times$ 10$^{-5}$ M.)
\end{enumerate}

Prepare the solutions given in the following table in dry beakers except the KI solution, which should be measured in a separate beaker. Initiate the reaction by pouring the KI solution suddenly into the mixture and start the stopwatch. You can do the ten experiments necessary relatively quickly if you prepare all the necessary samples and intiate them in one-minute intervals. Record the time at which the violet color of the iodine starch complex suddenly appears for each experiment.

After the first series of measurements, repeat experiments 1, 8, 9, and 10 but use distilled water instead of the KIO$_3$ solution. Measure the pH of these samples with a pH-meter calibrated using two buffers and calculate the hydrogen ion concentrations.

\begin{table}[H]
\caption{Composition of individual experiments}
\begin{tabular}{|c|c|c|c|c|c|c|c|}
\hline
Sample& KI & KIO$_3$ & H$_3$AsO$_3$ & starch & H$_2$O & Buffer A & Buffer B \\

number & cm$^3$ & cm$^3$ & cm$^3$ & cm$^3$ & cm$^3$ & cm$^3$ & cm$^3$ \\

\hline
1 & 6.0 & 2.0 & 0.5 & 1.0 & 7.5 & 33 & - \\

\hline
2 & 6.0 & 3.0 & 0.5 & 1.0 & 6.5 & 33 & - \\

\hline
3 & 6.0 & 4.0 & 0.5 & 1.0 & 5.5 & 33 & - \\

\hline
4 & 6.0 & 5.0 & 0.5 & 1.0 & 4.5 & 33 & - \\

\hline
5 & 8.0 & 2.0 & 0.5 & 1.0 & 5.5 & 33 & - \\

\hline
6 & 10.0 & 2.0 & 0.5 & 1.0 & 3.5 & 33 & - \\

\hline
7 & 12.5 & 2.0 & 0.5 & 1.0 & 1.0 & 33 & - \\

\hline
8 & 6.0 & 2.0 & 0.5 & 1.0 & 7.5 & 22 & 11 \\

\hline
9 & 6.0 & 2.0 & 0.5 & 1.0 & 7.5 & 11 & 22 \\

\hline
10 & 6.0 & 2.0 & 0.5 & 1.0 & 7.5 & - & 33 \\

\hline

\end{tabular}
\end{table}

\subsection{Evaluation}

Give the experimentally measured reaction times and the calculated initial rates in the from of a table:

\begin{table}[H]
\begin{tabular}{|c|c|c|c|}
\hline
Number & $t$ & $r_0$ & log$_{10} r_0$ \\

   & s & mol dm$^{-3}$ s$^{-1}$ &  \\

\hline
1 &   &   &   \\

\hline
2 &   &   &   \\

\hline
... &   &   &   \\

\hline
\end{tabular}
\end{table}

To find the individual orders of reaction, use the following series of data: measurements 1, 2, 3, and 4 for iodate ion dependence; measurements 1, 5, 6, and 7 for iodide ion dependence; measurements 1, 8, 9, and 10 for hydrogen ion dependence.

In the usual power law kinetics, there is a linear relationship between the logarithms of the initial rates and the logarithms of the concentrations of the component studied. The order of reaction is given by the slope. For example, for iodide ions:

\begin{equation}
\text{log}_{10} r_0 = \text{log}_{10} k' + {\beta \text{I}^{-}} \text{log}_{10} [\text{I}^{-}]
\end{equation}

The intercept of the fitted straight line is $k'$, it contains the product of the orders of reactions and initial concentrations of the remaining components and the value of the rate constant. Enumerate your results in the following tabular form:

\begin{table}[H]
\begin{tabular}{|c|c|c|c|c|c|c|c|}
\hline
Number & log$_{10} r_0$ & $[\text{IO}_{3}^{-}]$ & log$_{10}[\text{IO}_{3}^{-}]$ & $[\text{I}^{-}]$ & log$_{10}[\text{I}^{-}]$ & $[\text{H}^{+}]$ & log$_{10}[\text{H}^{+}]$ \\

   &  & mol dm$^{-3}$ &  & mol dm$^{-3}$ &  & mol dm$^{-3}$ &  \\

\hline
 &  &  &  &  &  &  &  \\

\hline
&  &  &  &  &  &  &  \\

\hline
&  &  &  &  &  &  &  \\

\hline
\end{tabular}
\end{table}

Plot $r_0$ as a function of the appropriate concentration and determine the individual reaction orders.
